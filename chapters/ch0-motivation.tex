\section{Motivation}
% Main points:
    % High overview of work.
    % Describe that have spent two years building this system.
    % Building a system with spin and motion control.
    % List of unique requirements for the system
        % Motional control
        % Single addressing
        % Standing waves
    % Here characterisation
    % But we want to use this experiment for...
        % E.g. Feynman diagram simulation
        % Here I would like to briefly describe the next steps in the project, as well
        % as the future experiments we want to demonstrate.
        % Discussion of single addressing could be here.

    The following chapter details the characterisation work performed during the
    commisioning of a new ion trap apparatus. Over the past two years, my
    colleagues and I have been constructing this system, which is now
    approaching readiness for our desired experiments. In this
    \textit{Motivation} section, I will provide a high-level overview of the
    physics we want to explore, state the unique aspects of the apparatus that
    enable these experiments, and finally, give a timeline for planned future
    work. The complexity of our system neccesitates collaboration, and so I
    would like to acknowledge the rest of the \emph{FastGates} team, for their
    extensive contributions to bringing the system online. \\

    There are two research directions that we are pursuing with this apparatus.
    Both involve the study and interplay between the ``spin'' electronic degree
    of freedom and the motion of the ion which we describe by a quantum harmonic
    oscillator.\\
    The first is the demonstration of fast and high-fidelity two-qubit
    entangling gates between trapped ions. High fidelity entangling gates are
    critical for developing a quantum computer which can perform useful
    calculations with high success probability~\cite{}. Gate errors associated with
    decoherence of the quantum state are often dominated by incoherent
    processes, such as spontaneous emission, or interactions between the qubit
    and the environment~\cite{wineland_experimental_1998}\mccorrect{XXX check relevance of citation}. These errors accumulate with time, and so may be
    mitigated by increasing the algorithm ``clock-speed'', hence our desire for
    fast entanglement~\cite{}. Typically in ion traps, local two-qubit gates are
    performed via the ion's shared motion, which may be excited by interaction
    with a laser~\cite{sorensen_entanglement_2000}. As the effective gate frequency approaches that of the
    frequency separation of the motional modes, the dynamics between spin and
    motion become complicated by multiple simultaneous mode excitations~\cite{schafer_fast_2018}. Our
    apparatus is designed to provide the required laser intensities and mode
    frequencies such that we can explore this regime, and demonstrate the
    required control to tame these multiple mode excitations and perform
    high-fidelity entangling gates.\\
    The second direction is the use of the motion of the ion directly as a
    quantum resource~\cite{}\mccorrect{XXX cite review papers here, zheng 2021}. This falls into the continuous variable quantum computing
    paradigm (CVQC), where instead of discrete qubits, the quantum state is
    described by bosonic ``qumodes''~\cite{}. CVQC reduces the
    technical overhead of scaling discrete systems, as larger Hilbert spaces are
    easily accessed due to the many level nature of the individual qumodes.
    We want to demonstrate a
    universal set of gates for CVQC, which includes, single-mode Gaussian and
    non-Gaussian gates (demonstrated in~\cite{bazavan_squeezing_2024}), and two-mode entangling gates. Further, we want to demonstrate the ability to simultaneously leverage both
    bosonic and spin degrees of freedom of our ions to efficiently encode
    problems in the full hybrid system we have access to~\cite{varona_towards_2024, brenner_factoring_2025}. Our apparatus has both
    precise motional control and stability through our trap choice, and the
    selective addressing system that is required when interacting with many-ion
    chains.\\
    
    Details of the apparatus are presented in the Appendices. Summarised here are the unique aspects enabling the desired experiments:
    \begin{itemize}
    \item \textbf{High laser intensities:} The system is designed to provide
        high laser intensities, which are required to drive fast entangling
        gates. This is achieved through the use of a 5~W Ti:Sapph system (section~\ref{sec:Narrow Line Width 729 Laser}) and
        two high numerical aperture (NA = 0.6) objectives (section~\ref{sec:Vacuum System}).
    \item \textbf{Single ion addressing:} Using crossed acousto-optic
        deflectors, we can selectively address individual ions in a multi-ion chain (section~\ref{sec:Single Addressing System}).
        This enables fast entangling gates between ions in a long
        chain, and the ability to perform motional interactions on selected modes.
    \item \textbf{Standing wave addressing:} The system is designed to provide
        standing optical waves at the ion position. Standing waves are required for
        some fast gate schemes~\cite{saner_breaking_2023}. 
    \item \textbf{Motional control:} The Paul trap used in the system, produced
        by \emph{National Physical Laboratory} in the UK, has low heating rates (section~\ref{sec:Trap Heating Rates}),
        deep trapping potentials relative to surface style ion traps, and flexible
        control over DC confinement (section~\ref{sec:The Ion Trap}). These features ensure that we have modes
        with long coherence times (section~\ref{sec:Motional Mode Stability}), making them suitable for encoding quantum
        information.
    \end{itemize}

    Here, the characterisation of the spin qubits is in the context of discrete
    variable quantum computing. We are interested in the fidelity of qubit
    state-preparation, measurement (section~\ref{sec:SPAM}), single- (section~\ref{sec:Randomised Benchmarking}) and two-qubit gates (section~\ref{sec:Two-Qubit Entangling Gates}). These
    individual elements comprise longer circuits, and so all contribute to the
    overall fidelity of any chosen algorithm.\\
    The motional characterisation is currently focused on the stability of the
    modes (section~\ref{sec:Motional Mode Stability}), and the ability to perform spin-dependent forces, SDFs (section~\ref{sec:Spin-Dependent Forces}). SDFs are the
    base interaction on which many of the desired CVQC gates are
    comprised~\cite{sutherland_universal_2021}, and so these must be well
    characterised and understood in our system.\\

    To push the system forwards, in the next 6~months, we will focus on
    commisioning the single addressing system, and on demonstrating squeezing of
    the motional state of the ion.\\
    As described above, the single addressing system is a critical component for
    many of our planned experiments. Section~\ref{sec:Single Addressing System},
    and~\cite{oevergaard_thesis} describe the system in detail. The next
    immediate steps are to align the constructed optical system to the ion
    chain, and to implement the logical and RF control system required to drive
    the acousto-optic deflectors. Once this is complete, we will characterise
    the cross-talk of the system by performing single addressing of ions in a
    chain and measuring spin populations. We will then be able to demonstrate
    fast entangling gates between ions in a chain, and the ability to perform
    motional interactions on selected modes.\\
    Squeezing of the motional mode via the method outlined
    in~\cite{sutherland_universal_2021}, is a significant step towards
    demonstrating the universal set of gates for CVQC. This is due to
    the constituent interactions of the effective squeezing gate (two SDFs), are
    the same interactions needed for trisqueezing~\cite{braunstein_generalised_1987, bazavan_squeezing_2024} --- a non-Gaussian gate, and
    for two-mode squeezing~\cite{lvovsky_squeezed_2016}\mccorrect{XXX tempt fate and cite Lvovsky} --- an entangling gate. Furthermore, once we have
    demonstrated generating squeezing characteristic functions, there is a clear
    path to experimentally demonstrating the extraction of Feynman diagram terms
    with the hybrid system being used as a quantum
    simulator~\cite{varona_towards_2024}.\\
