\documentclass[12pt]{report}
\usepackage{graphicx}

\begin{document}

% ------------------------------------------------------------------------

\chapter{Introduction}

% Here I would like a brief motivation for the work.
% Perhaps also a bit of theory goes here?

% ------------------------------------------------------------------------

\chapter{Ion Trap Apparatus}

A vast effort is spent on the initial build-up of the an ion trap system, but
throughout the life of the experiment, a greater effort is spent on its daily
maintenance.  I hope that this chapter will serve as a resource for future
members of the FastGates team, as well as provide a useful recipe for anyone
building a similar system. \\

Due to the size and complexity of the system, we introduce an inital overview of
the design, motivated by the desired functions.  As the name suggests, an ion
trap experiment aims to confine arrays of single ions, this is achieved by
static and dynamic electric fields which, due to the ions possesing non-zero
electric charge, can provide trapping potentials, section~\ref{sec:The Ion
Trap}. Due to the fragility of the internal states of the ion (these are state
of the art sensors after all), we must take great care in isolating the ion from
any noisy environment. This neccesitates the use of ultra-high vacuum (UHV)
systems, section~\ref{sec:Vacuum System}, vibration isolation, and magnetic
shielding, section~\ref{sec:System Design}. To manipulate the internal electronic states of the ion, we create
local electric and magnetic fields using RF antennae and, in this work, lasers,
sections~\ref{sec:Laser systems} and~\ref{sec:Narrow Line Width 729 Laser}.
Finally, to interface with the apparatus we have built, at the time scales set by our interaction strengths, we require a sophisticated and custom control system which is discussed in section~\ref{sec:Sinara Hardware}.

\section{System Design}
\label{sec:System Design}
% figures for section
    % Block diagram of the system
    % Calcium level structure 
    % Solidworks models

\section{The Ion Trap}
\label{sec:The Ion Trap}
% figures for section
    % NPL trap with relevant size scales

% Potentially this lives in characterisation?
\subsection{Trap RF Chain}
\subsection{Trap DC Voltages}

\section{Vacuum System and Beam access}
\label{sec:Vacuum System}
% figures for section
    % Technical drawing of beam geometries and magnetic field to ion chain.
    % Drawing of ion trap package with high NA lenses

\section{Laser systems}
\label{sec:Laser systems}
% figures for section

% All frequencies and powers and how we control these. PDH lock, AOM etc.

\section{Narrow Line Width 729 Laser}
\label{sec:Narrow Line Width 729 Laser} 
% figures for section
    % Solstis cavity
    % Beam path and control loops

% FNC

\subsection{Single Addressing System}
\label{sec:Single Addressing System}
% figures for section

\section{Sinara Hardware and Artiq}
\label{sec:Sinara Hardware}
% figures for section

% ------------------------------------------------------------------------

\chapter{Experiment Characterisation}

Before we can dive into running novel experiments involving the motion and spin
of the atoms, we need to characterise our apparatus. This allows us to both
benchmark our system against state of the art results, and to reveal any
current limitations of the apparatus which we may need to address.

% ------------------------------------------------------------------------

\chapter{Outlook}

% Here I would like to briefly describe the next steps in the project, as well
% as the future expeiments we want to demonstrate.

% ------------------------------------------------------------------------

\clearpage


\clearpage
\section{Appendix}

\end{document}